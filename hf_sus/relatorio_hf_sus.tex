% Options for packages loaded elsewhere
\PassOptionsToPackage{unicode}{hyperref}
\PassOptionsToPackage{hyphens}{url}
\PassOptionsToPackage{dvipsnames,svgnames,x11names}{xcolor}
%
\documentclass[
  letterpaper,
  DIV=11,
  numbers=noendperiod]{scrartcl}

\usepackage{amsmath,amssymb}
\usepackage{lmodern}
\usepackage{iftex}
\ifPDFTeX
  \usepackage[T1]{fontenc}
  \usepackage[utf8]{inputenc}
  \usepackage{textcomp} % provide euro and other symbols
\else % if luatex or xetex
  \usepackage{unicode-math}
  \defaultfontfeatures{Scale=MatchLowercase}
  \defaultfontfeatures[\rmfamily]{Ligatures=TeX,Scale=1}
\fi
% Use upquote if available, for straight quotes in verbatim environments
\IfFileExists{upquote.sty}{\usepackage{upquote}}{}
\IfFileExists{microtype.sty}{% use microtype if available
  \usepackage[]{microtype}
  \UseMicrotypeSet[protrusion]{basicmath} % disable protrusion for tt fonts
}{}
\makeatletter
\@ifundefined{KOMAClassName}{% if non-KOMA class
  \IfFileExists{parskip.sty}{%
    \usepackage{parskip}
  }{% else
    \setlength{\parindent}{0pt}
    \setlength{\parskip}{6pt plus 2pt minus 1pt}}
}{% if KOMA class
  \KOMAoptions{parskip=half}}
\makeatother
\usepackage{xcolor}
\usepackage[normalem]{ulem}
\setlength{\emergencystretch}{3em} % prevent overfull lines
\setcounter{secnumdepth}{-\maxdimen} % remove section numbering
% Make \paragraph and \subparagraph free-standing
\ifx\paragraph\undefined\else
  \let\oldparagraph\paragraph
  \renewcommand{\paragraph}[1]{\oldparagraph{#1}\mbox{}}
\fi
\ifx\subparagraph\undefined\else
  \let\oldsubparagraph\subparagraph
  \renewcommand{\subparagraph}[1]{\oldsubparagraph{#1}\mbox{}}
\fi

\usepackage{color}
\usepackage{fancyvrb}
\newcommand{\VerbBar}{|}
\newcommand{\VERB}{\Verb[commandchars=\\\{\}]}
\DefineVerbatimEnvironment{Highlighting}{Verbatim}{commandchars=\\\{\}}
% Add ',fontsize=\small' for more characters per line
\usepackage{framed}
\definecolor{shadecolor}{RGB}{241,243,245}
\newenvironment{Shaded}{\begin{snugshade}}{\end{snugshade}}
\newcommand{\AlertTok}[1]{\textcolor[rgb]{0.68,0.00,0.00}{#1}}
\newcommand{\AnnotationTok}[1]{\textcolor[rgb]{0.37,0.37,0.37}{#1}}
\newcommand{\AttributeTok}[1]{\textcolor[rgb]{0.40,0.45,0.13}{#1}}
\newcommand{\BaseNTok}[1]{\textcolor[rgb]{0.68,0.00,0.00}{#1}}
\newcommand{\BuiltInTok}[1]{\textcolor[rgb]{0.00,0.23,0.31}{#1}}
\newcommand{\CharTok}[1]{\textcolor[rgb]{0.13,0.47,0.30}{#1}}
\newcommand{\CommentTok}[1]{\textcolor[rgb]{0.37,0.37,0.37}{#1}}
\newcommand{\CommentVarTok}[1]{\textcolor[rgb]{0.37,0.37,0.37}{\textit{#1}}}
\newcommand{\ConstantTok}[1]{\textcolor[rgb]{0.56,0.35,0.01}{#1}}
\newcommand{\ControlFlowTok}[1]{\textcolor[rgb]{0.00,0.23,0.31}{#1}}
\newcommand{\DataTypeTok}[1]{\textcolor[rgb]{0.68,0.00,0.00}{#1}}
\newcommand{\DecValTok}[1]{\textcolor[rgb]{0.68,0.00,0.00}{#1}}
\newcommand{\DocumentationTok}[1]{\textcolor[rgb]{0.37,0.37,0.37}{\textit{#1}}}
\newcommand{\ErrorTok}[1]{\textcolor[rgb]{0.68,0.00,0.00}{#1}}
\newcommand{\ExtensionTok}[1]{\textcolor[rgb]{0.00,0.23,0.31}{#1}}
\newcommand{\FloatTok}[1]{\textcolor[rgb]{0.68,0.00,0.00}{#1}}
\newcommand{\FunctionTok}[1]{\textcolor[rgb]{0.28,0.35,0.67}{#1}}
\newcommand{\ImportTok}[1]{\textcolor[rgb]{0.00,0.46,0.62}{#1}}
\newcommand{\InformationTok}[1]{\textcolor[rgb]{0.37,0.37,0.37}{#1}}
\newcommand{\KeywordTok}[1]{\textcolor[rgb]{0.00,0.23,0.31}{#1}}
\newcommand{\NormalTok}[1]{\textcolor[rgb]{0.00,0.23,0.31}{#1}}
\newcommand{\OperatorTok}[1]{\textcolor[rgb]{0.37,0.37,0.37}{#1}}
\newcommand{\OtherTok}[1]{\textcolor[rgb]{0.00,0.23,0.31}{#1}}
\newcommand{\PreprocessorTok}[1]{\textcolor[rgb]{0.68,0.00,0.00}{#1}}
\newcommand{\RegionMarkerTok}[1]{\textcolor[rgb]{0.00,0.23,0.31}{#1}}
\newcommand{\SpecialCharTok}[1]{\textcolor[rgb]{0.37,0.37,0.37}{#1}}
\newcommand{\SpecialStringTok}[1]{\textcolor[rgb]{0.13,0.47,0.30}{#1}}
\newcommand{\StringTok}[1]{\textcolor[rgb]{0.13,0.47,0.30}{#1}}
\newcommand{\VariableTok}[1]{\textcolor[rgb]{0.07,0.07,0.07}{#1}}
\newcommand{\VerbatimStringTok}[1]{\textcolor[rgb]{0.13,0.47,0.30}{#1}}
\newcommand{\WarningTok}[1]{\textcolor[rgb]{0.37,0.37,0.37}{\textit{#1}}}

\providecommand{\tightlist}{%
  \setlength{\itemsep}{0pt}\setlength{\parskip}{0pt}}\usepackage{longtable,booktabs,array}
\usepackage{calc} % for calculating minipage widths
% Correct order of tables after \paragraph or \subparagraph
\usepackage{etoolbox}
\makeatletter
\patchcmd\longtable{\par}{\if@noskipsec\mbox{}\fi\par}{}{}
\makeatother
% Allow footnotes in longtable head/foot
\IfFileExists{footnotehyper.sty}{\usepackage{footnotehyper}}{\usepackage{footnote}}
\makesavenoteenv{longtable}
\usepackage{graphicx}
\makeatletter
\def\maxwidth{\ifdim\Gin@nat@width>\linewidth\linewidth\else\Gin@nat@width\fi}
\def\maxheight{\ifdim\Gin@nat@height>\textheight\textheight\else\Gin@nat@height\fi}
\makeatother
% Scale images if necessary, so that they will not overflow the page
% margins by default, and it is still possible to overwrite the defaults
% using explicit options in \includegraphics[width, height, ...]{}
\setkeys{Gin}{width=\maxwidth,height=\maxheight,keepaspectratio}
% Set default figure placement to htbp
\makeatletter
\def\fps@figure{htbp}
\makeatother

\usepackage{booktabs}
\usepackage{longtable}
\usepackage{array}
\usepackage{multirow}
\usepackage{wrapfig}
\usepackage{float}
\usepackage{colortbl}
\usepackage{pdflscape}
\usepackage{tabu}
\usepackage{threeparttable}
\usepackage{threeparttablex}
\usepackage[normalem]{ulem}
\usepackage{makecell}
\usepackage{xcolor}
\KOMAoption{captions}{tableheading}
\makeatletter
\makeatother
\makeatletter
\makeatother
\makeatletter
\@ifpackageloaded{caption}{}{\usepackage{caption}}
\AtBeginDocument{%
\ifdefined\contentsname
  \renewcommand*\contentsname{Table of contents}
\else
  \newcommand\contentsname{Table of contents}
\fi
\ifdefined\listfigurename
  \renewcommand*\listfigurename{List of Figures}
\else
  \newcommand\listfigurename{List of Figures}
\fi
\ifdefined\listtablename
  \renewcommand*\listtablename{List of Tables}
\else
  \newcommand\listtablename{List of Tables}
\fi
\ifdefined\figurename
  \renewcommand*\figurename{Figure}
\else
  \newcommand\figurename{Figure}
\fi
\ifdefined\tablename
  \renewcommand*\tablename{Table}
\else
  \newcommand\tablename{Table}
\fi
}
\@ifpackageloaded{float}{}{\usepackage{float}}
\floatstyle{ruled}
\@ifundefined{c@chapter}{\newfloat{codelisting}{h}{lop}}{\newfloat{codelisting}{h}{lop}[chapter]}
\floatname{codelisting}{Listing}
\newcommand*\listoflistings{\listof{codelisting}{List of Listings}}
\makeatother
\makeatletter
\@ifpackageloaded{caption}{}{\usepackage{caption}}
\@ifpackageloaded{subcaption}{}{\usepackage{subcaption}}
\makeatother
\makeatletter
\@ifpackageloaded{tcolorbox}{}{\usepackage[many]{tcolorbox}}
\makeatother
\makeatletter
\@ifundefined{shadecolor}{\definecolor{shadecolor}{rgb}{.97, .97, .97}}
\makeatother
\makeatletter
\makeatother
\ifLuaTeX
  \usepackage{selnolig}  % disable illegal ligatures
\fi
\IfFileExists{bookmark.sty}{\usepackage{bookmark}}{\usepackage{hyperref}}
\IfFileExists{xurl.sty}{\usepackage{xurl}}{} % add URL line breaks if available
\urlstyle{same} % disable monospaced font for URLs
\hypersetup{
  pdftitle={HF - SUS},
  pdfauthor={Danilo Imbimbo, Pedro Buril, Raulino Sabino},
  colorlinks=true,
  linkcolor={blue},
  filecolor={Maroon},
  citecolor={Blue},
  urlcolor={Blue},
  pdfcreator={LaTeX via pandoc}}

\title{HF - SUS}
\author{Danilo Imbimbo, Pedro Buril, Raulino Sabino}
\date{}

\begin{document}
\maketitle
\ifdefined\Shaded\renewenvironment{Shaded}{\begin{tcolorbox}[frame hidden, boxrule=0pt, breakable, interior hidden, sharp corners, enhanced, borderline west={3pt}{0pt}{shadecolor}]}{\end{tcolorbox}}\fi

\hypertarget{estimativa-de-cuxe1lculo-do-hf.1.1.1-sus}{%
\section{Estimativa de Cálculo do HF.1.1.1
(SUS)}\label{estimativa-de-cuxe1lculo-do-hf.1.1.1-sus}}

\hypertarget{introduuxe7uxe3o}{%
\subsection{Introdução}\label{introduuxe7uxe3o}}

Esse documento tem como objetivo realizar uma estimativa do HF.1.1.1
(Sistema Único de Saúde) por meio da metodologia do System of Health
Accounts, com base nos documentos encontrados acerca do SHA brasileiro e
no que foi possível recuperar da metodologia construída pelo grupo de
contas.

A HF.1.1.1 se refere ao esquema de financiamento (HF) do Sistema Único
de Saúde (SUS), sistema universal que possui como público-alvo toda a
população brasileira. O SUS é o esquema de financiamento mais relevante
em termos de volume de recursos. Seu financiamento é tripartite, ou
seja, financiado por União, Estados e Municípios, havendo uma legislação
que estabelece os investimentos mínimos em saúde por cada esfera de
governo. Suas receitas provêm majoritariamente por meio de tributos.
Todavia, além disso, dispõe de transferências de governos estrangeiros,
doações, bem como outras receitas domésticas.

Dessa forma, consideram-se as despesas SUS aquelas que se referem ao
atendimento universal, e não a todo gasto público de saúde. Essa
diferenciação é relevante, pois, existem regimes de financiamento
públicos não universais, como os gastos com saúde dos servidores civis e
militares. Assim, se encaixam nessa lógica as despesas efetuadas por
Ministério da Saúde (União), Estados e Municípios, sendo as despesas da
União compostas pelas despesas do Ministério da Saúde e do Ministério da
Educação (MEC) com Hospitais Universitários, que além de centros de
formação de recursos humanos, prestam assistência ao SUS, sendo centros
de referência de média e alta complexidade. Esses hospitais são geridos
pela Empresa Brasileira de Serviços Hospitalares (Ebserh), vinculada ao
MEC.

Destarte, esta nota técnica apresentará os métodos e técnicos realizados
para a estimativa das despesas com a HF.1.1.1 para as três esferas de
governo, somadas as despesas do MEC, que se somam as despesas da União.

\hypertarget{fontes-de-dados}{%
\subsection{Fontes de Dados}\label{fontes-de-dados}}

Esse trabalho utiliza os dados extraídos do Sistema Integrado de
Administração Financeira (SIAFI), considerando a execução orçamentária
do Ministério da Saúde do Ministério da Educação (Hospitais
Universitários), bem como os dados do Sistema de Informações sobre
Orçamentos Públicos em Saúde.

O SIAFI é o principal instrumento para registro, acompanhamento e
controle da execução orçamentária do governo federal, contendo
informações sobre toda a Administração Pública Direta federal, além de
autarquias, fundações, empresas públicas e sociedades de economia mista,
considerando o Orçamento Fical ou Orçamento da Seguridade Social. Por
outro lado, o SIOPS contempla os registros de receitas e despesas em
todos os entes federados.

Os dados do SIAFI são extraídos pelo Tesouro Gerencial em relatórios
.xlsx a partir dos filtros selecionados. No caso, foi extraído um
relatório a partir da Função 10, que consolida as despesas classificadas
como Saúde.

No caso do SIOPS, foram solicitados os dados para a equipe responsável,
a Coordenação de Informações sobre Orçamentos Públicos em Saúde
(CSIOPS), da Coordenação-Geral de Informações em Economia da Saúde, do
Departamento de Economia e Desenvolvimento da Saúde (CGES/DESID), de
acordo com a estrutura regimental do Ministério da Saúde aprovada por
meio do Decreto Nº 11.798, de 28 de Novembro de 2023. Esses dados foram
encaminhados também em arquivos .xlsx.

\hypertarget{running-code}{%
\subsection{Running Code}\label{running-code}}

A partir daqui, serão apresentados os códigos gerados para a realização
das estimações.

\hypertarget{limpeza-de-ambiente-e-carregamento-dos-pacotes}{%
\subsubsection{Limpeza de Ambiente e carregamento dos
pacotes}\label{limpeza-de-ambiente-e-carregamento-dos-pacotes}}

São utilizados sete principais pacotes para essa estimativa: readxl,
dplyr, tidyr, knitr, kableExtra, zoo, reshape2 e ggplot2.

\begin{Shaded}
\begin{Highlighting}[]
\FunctionTok{gc}\NormalTok{()}
\end{Highlighting}
\end{Shaded}

\begin{verbatim}
          used (Mb) gc trigger (Mb) max used (Mb)
Ncells  560659 30.0    1279791 68.4   644182 34.5
Vcells 1003871  7.7    8388608 64.0  1634924 12.5
\end{verbatim}

\begin{Shaded}
\begin{Highlighting}[]
\FunctionTok{rm}\NormalTok{(}\AttributeTok{list=}\FunctionTok{ls}\NormalTok{())}

\FunctionTok{print}\NormalTok{(}\StringTok{"Ambiente Limpo"}\NormalTok{)}
\end{Highlighting}
\end{Shaded}

\begin{verbatim}
[1] "Ambiente Limpo"
\end{verbatim}

\begin{Shaded}
\begin{Highlighting}[]
\FunctionTok{library}\NormalTok{(readxl)}
\FunctionTok{library}\NormalTok{(dplyr)}
\FunctionTok{library}\NormalTok{(tidyr)}
\FunctionTok{library}\NormalTok{(knitr)}
\FunctionTok{library}\NormalTok{(kableExtra)}
\FunctionTok{library}\NormalTok{(zoo)}
\FunctionTok{library}\NormalTok{(reshape2)}
\FunctionTok{library}\NormalTok{(ggplot2)}
\end{Highlighting}
\end{Shaded}

\hypertarget{importando-a-base-de-dados-do-siafi}{%
\subsubsection{Importando a base de dados do
SIAFI}\label{importando-a-base-de-dados-do-siafi}}

\begin{Shaded}
\begin{Highlighting}[]
\NormalTok{diretorio }\OtherTok{\textless{}{-}} \FunctionTok{getwd}\NormalTok{()}
\NormalTok{lista\_dfs }\OtherTok{\textless{}{-}} \FunctionTok{list}\NormalTok{()}

\NormalTok{arquivos }\OtherTok{\textless{}{-}} \FunctionTok{list.files}\NormalTok{(diretorio, }\AttributeTok{pattern =} \StringTok{"}\SpecialCharTok{\textbackslash{}\textbackslash{}}\StringTok{.xlsx$"}\NormalTok{, }\AttributeTok{full.names =} \ConstantTok{TRUE}\NormalTok{)}
\ControlFlowTok{for}\NormalTok{ (arquivo }\ControlFlowTok{in}\NormalTok{ arquivos) \{}
  \CommentTok{\# Ler o arquivo xlsx}
\NormalTok{  df }\OtherTok{\textless{}{-}} \FunctionTok{read\_excel}\NormalTok{(arquivo)}
  
  \CommentTok{\# Determinar a origem do orçamento}
  \ControlFlowTok{if}\NormalTok{ (}\FunctionTok{grepl}\NormalTok{(}\StringTok{"EDUCACAO"}\NormalTok{, }\FunctionTok{basename}\NormalTok{(arquivo), }\AttributeTok{ignore.case =} \ConstantTok{TRUE}\NormalTok{)) \{}
\NormalTok{    origem }\OtherTok{\textless{}{-}} \StringTok{"EDUCACAO"}
\NormalTok{  \} }\ControlFlowTok{else} \ControlFlowTok{if}\NormalTok{ (}\FunctionTok{grepl}\NormalTok{(}\StringTok{"SAUDE"}\NormalTok{, }\FunctionTok{basename}\NormalTok{(arquivo), }\AttributeTok{ignore.case =} \ConstantTok{TRUE}\NormalTok{)) \{}
\NormalTok{    origem }\OtherTok{\textless{}{-}} \StringTok{"SAUDE"}
\NormalTok{  \} }\ControlFlowTok{else}\NormalTok{ \{}
\NormalTok{    origem }\OtherTok{\textless{}{-}} \StringTok{"OUTRO"}
\NormalTok{  \}}
   
  \CommentTok{\# Adicionar a coluna ORIGEM\_ORÇAMENTO}
\NormalTok{  df }\OtherTok{\textless{}{-}} \FunctionTok{mutate}\NormalTok{(df, ORIGEM\_ORÇAMENTO }\OtherTok{=}\NormalTok{ origem)}
  
  \CommentTok{\# Adicionar o data frame modificado à lista}
\NormalTok{  lista\_dfs[[}\FunctionTok{length}\NormalTok{(lista\_dfs) }\SpecialCharTok{+} \DecValTok{1}\NormalTok{]] }\OtherTok{\textless{}{-}}\NormalTok{ df}
\NormalTok{\}}

\CommentTok{\# Concatenar todos os data frames em um único data frame}
\NormalTok{df\_final }\OtherTok{\textless{}{-}} \FunctionTok{bind\_rows}\NormalTok{(lista\_dfs)}

\FunctionTok{rm}\NormalTok{(df, lista\_dfs)}
\end{Highlighting}
\end{Shaded}

\hypertarget{realizauxe7uxe3o-de-teste-de-verificauxe7uxe3o}{%
\subsubsection{Realização de teste de
verificação}\label{realizauxe7uxe3o-de-teste-de-verificauxe7uxe3o}}

Realização de testes na base de dados a fim de verificar se a extração e
importação estão corretos.

\begin{Shaded}
\begin{Highlighting}[]
\NormalTok{teste\_despesas }\OtherTok{\textless{}{-}}\NormalTok{ df\_final }\SpecialCharTok{\%\textgreater{}\%} 
  \FunctionTok{group\_by}\NormalTok{(}\StringTok{\textasciigrave{}}\AttributeTok{Ano Lançamento}\StringTok{\textasciigrave{}}\NormalTok{, ORIGEM\_ORÇAMENTO) }\SpecialCharTok{\%\textgreater{}\%} 
  \FunctionTok{summarise}\NormalTok{(}\FunctionTok{sum}\NormalTok{(}\StringTok{\textasciigrave{}}\AttributeTok{DESPESAS PAGAS}\StringTok{\textasciigrave{}}\NormalTok{, }\AttributeTok{na.rm =}\NormalTok{ T))}
\end{Highlighting}
\end{Shaded}

\begin{verbatim}
`summarise()` has grouped output by 'Ano Lançamento'. You can override using
the `.groups` argument.
\end{verbatim}

\begin{Shaded}
\begin{Highlighting}[]
\FunctionTok{rm}\NormalTok{(teste\_despesas)}

\FunctionTok{gc}\NormalTok{()}
\end{Highlighting}
\end{Shaded}

\begin{verbatim}
          used (Mb) gc trigger  (Mb) max used  (Mb)
Ncells  997746 53.3    1847141  98.7  1847141  98.7
Vcells 8248748 63.0   18972651 144.8 16612682 126.8
\end{verbatim}

\hypertarget{calculando-o-gasto-mec-com-sauxfade}{%
\subsubsection{Calculando o gasto MEC com
saúde}\label{calculando-o-gasto-mec-com-sauxfade}}

Entendendo a necessidade de consolidação das despesas com Hospitais
Universitários, a execução orçamentária do Ministério da Educação (MEC)
é essencial. Essas despesas são alocadas no MEC. Dessa forma, essas
despesas são estimadas da seguinte forma:

Considerou-se o montante de despesas da Função Educação (Função 12)
identificado como despesas correntes (Categoria Econômica Despesa = 3) e
realizadas pelas Unidades Gestoras relacionadas aos estabelecimentos de
saúde das Universidades Federais (Tabela 1), bem como as despesas nas
ações 20XR e 4086. A ação 20XR se refere ao ``Funcionamento de
Instituições Federais de Ensino Superios'' e a 4086 ao ``Funcionamento e
Gestão de Instituições Hospitalares Federais''.

\begin{Shaded}
\begin{Highlighting}[]
\NormalTok{HUs\_ }\OtherTok{\textless{}{-}}\NormalTok{ df\_final }\SpecialCharTok{\%\textgreater{}\%}
  \FunctionTok{filter}\NormalTok{(UO\_Orgao\_Maximo }\SpecialCharTok{==} \StringTok{"MINISTERIO DA EDUCACAO"}\NormalTok{) }\SpecialCharTok{\%\textgreater{}\%} 
  \FunctionTok{select}\NormalTok{(}\StringTok{"UG\_Executora\_COD"}\NormalTok{, }\StringTok{"UG\_Executora"}\NormalTok{) }\SpecialCharTok{\%\textgreater{}\%}
  \FunctionTok{filter}\NormalTok{(}\FunctionTok{grepl}\NormalTok{(}\StringTok{"HOSP|COMPLEXO HOSP|EBSERH|MATERNIDADE|DOENCAS DO TORAX DA UFRJ|GINECOLOGIA DA UFRJ|DEOLINDO COUTO DA UFRJ|PSIQUIATRIA DA UFRJ|PUERIC. PED MAT. GESTEIRA DA UFRJ|HUSM"}\NormalTok{, UG\_Executora) }\SpecialCharTok{\&} \SpecialCharTok{!}\FunctionTok{grepl}\NormalTok{(}\StringTok{"EXERCITO|MILITAR"}\NormalTok{, UG\_Executora)) }\SpecialCharTok{\%\textgreater{}\%}
  \FunctionTok{unique}\NormalTok{()}

\NormalTok{HUs\_renamed }\OtherTok{\textless{}{-}}\NormalTok{ HUs\_ }\SpecialCharTok{\%\textgreater{}\%} 
  \FunctionTok{rename}\NormalTok{(}\StringTok{\textquotesingle{}Código UF Executora\textquotesingle{}} \OtherTok{=}\NormalTok{ UG\_Executora\_COD, }\StringTok{\textquotesingle{}UG Executora\textquotesingle{}} \OtherTok{=}\NormalTok{ UG\_Executora)}

\FunctionTok{kable}\NormalTok{(HUs\_renamed, }\AttributeTok{caption =} \StringTok{"Tabela 1 {-} Lista de Hospitais Universitários"}\NormalTok{) }\SpecialCharTok{\%\textgreater{}\%}
  \FunctionTok{kable\_styling}\NormalTok{(}\AttributeTok{bootstrap\_options =} \FunctionTok{c}\NormalTok{(}\StringTok{"striped"}\NormalTok{, }\StringTok{"hover"}\NormalTok{, }\StringTok{"condensed"}\NormalTok{, }\StringTok{"responsive"}\NormalTok{), }
                \AttributeTok{full\_width =}\NormalTok{ F, }
                \AttributeTok{position =} \StringTok{"center"}\NormalTok{)}
\end{Highlighting}
\end{Shaded}

\begin{longtable}[t]{ll}
\caption{Tabela 1 - Lista de Hospitais Universitários}\\
\toprule
Código UF Executora & UG Executora\\
\midrule
153150 & INSTITUTO PUERIC. PED MAT. GESTEIRA DA UFRJ\\
153152 & HOSPITAL UNIVERSITARIO DA UFRJ\\
153155 & MATERNIDADE ESCOLA DA UFRJ\\
150233 & HOSPITAL DE CLINICAS DA UFU\\
154177 & HOSPITAL UNIVERSITARIO DA FUFSE\\
\addlinespace
150247 & COMPLEXO HOSPITALAR E DE SAUDE DA UFBA\\
153040 & COMPLEXO HOSP UNIVERS PROF EDGARD SANTOS\\
153057 & HOSPITAL UNIVERSITARIO ANTONIO PEDRO\\
150231 & HOSPITAL UNIVERSITARIO\\
153261 & HOSPITAL CLINICAS/UFMG\\
\addlinespace
153071 & HOSPITAL UNIVERSITARIO LAURO WANDERLEY-UFPB\\
153808 & HOSPITAL DE CLINICAS DA UFPR\\
150432 & SUP.GERAL DO COMPLEXO HOSP.E DE SAUDE DA UFRJ\\
153147 & INSTITUTO DE GINECOLOGIA DA UFRJ\\
153148 & INSTITUTO DE NEUROL. DEOLINDO COUTO DA UFRJ\\
\addlinespace
153149 & INSTITUTO DE PSIQUIATRIA DA UFRJ\\
153151 & INSTITUTO DE DOENCAS DO TORAX DA UFRJ\\
158220 & HOSPITAL ESCOLA SAO FRANCISCO DE ASSIS\\
153610 & UNIVERSIDADE FEDERAL DE SANTA MARIA - HUSM\\
158196 & HOSPITAL UNIVERSITARIO ALCIDES CARNEIRO/UFCG\\
\addlinespace
154035 & HOSPITAL UNIV. GAFFREE E GUINLE DA UNIRIO\\
150224 & HOSPITAL UNIVERSITARIO GETULIO VARGAS\\
154106 & HOSPITAL UNIVERSITARIO DE BRASILIA - HUB\\
154072 & HOSPITAL UNIVERSITARIO\\
154070 & HOSPITAL UNIVERSITARIO JULIO MULLER DA FUFMT\\
\addlinespace
154145 & HOSPITAL ESCOLA DA UFPEL\\
150237 & HOSPITAL UNIVERSITARIO DA UFPI\\
155001 & HOSPITAL DE CLINICAS DE PORTO ALEGRE\\
150248 & HOSPITAL UNIVERSITARIO (HU/UFGD)\\
155016 & EBSERH HU-UFGD\\
\addlinespace
150229 & HOSPITAL UNIVERSITARIO PROF ALBERTO ANTUNES\\
150223 & MATERNIDADE CLIMERIO DE OLIVEIRA\\
150244 & HOSPITAL UNIVERSITARIO WALTER CANTIDIO\\
150246 & MATERNIDADE ESCOLA ASSIS CHATEAUBRIAND\\
153047 & HOSPITAL UNIVERSITARIO C. ANTONIO MORAIS/UFES\\
\addlinespace
153054 & HOSPITAL DAS CLINICAS DA UFGO\\
158172 & HOSPITAL UNIVERSITARIO JOAO DE BARROS BARRETO\\
150220 & HOSPITAL UNIVERSITARIO BETTINA FERRO DE SOUZA\\
153094 & HOSPITAL DAS CLINICAS - UFPE\\
150426 & UFRN - COMPLEXO HOSPITALAR DE SAUDE\\
\addlinespace
150232 & HOSPITAL UNIVERSITARIO - UFSC\\
150221 & HOSPITAL DE CLINICAS DA UFTM\\
155008 & HOSPITAL UNIVERSITARIO DO PIAUI\\
155007 & EMPRESA BRASILEIRA DE SERVICOS HOSPITALARES\\
152477 & UNIFESP-HOSP. UNIVERSITARIO\\
\addlinespace
158705 & HOSPITAL UNIVERSITARIO JULIO M BANDEIRA MELLO\\
155009 & HOSPITAL UNIVERSITARIO DE BRASILIA\\
155010 & HOSPITAL UNIVERSITARIO DA UFMA\\
155014 & HOSPITAL UNIVERSITARIO ANA BEZERRA\\
155021 & HOSPITAL DAS CLINICAS DA UFMG\\
\addlinespace
155023 & HOSPITAL UNIVERSITARIO LAURO WANDERLEY\\
155124 & HOSP UNIVERSITARIO MARIA APARECIDA PEDROSSIAN\\
155125 & HOSPITAL UNIVERSITARIO DE SANTA MARIA\\
155180 & HOSP DE ENSINO DR WASHINGTON ANTONIO DE BARRO\\
155905 & HOSPITAL DE DOENCAS TROPICAIS\\
\addlinespace
155011 & HOSPITAL DE CLINICAS DA UFTM\\
155012 & HOSP UNIVERSITARIO CASSIANO ANTONIO DE MORAES\\
155013 & HOSPITAL UNIVERSITARIO ONOFRE LOPES\\
155015 & MATERNIDADE ESCOLA JANUARIO CICCO\\
155017 & HOSPITAL UNIVERSITARIO DA UFS\\
\addlinespace
155018 & HOSPITAL UNIVERSITARIO GETULIO VARGAS\\
155019 & HOSPITAL UNIVERSITARIO JULIO MULLER\\
155020 & EBSERH COMPLEXO HOSPITALAR DO CEARA\\
155022 & EBSERH HC-UFPE\\
155126 & HOSPITAL UNIVERSITARIO PROF ALBERTO ANTUNES\\
\addlinespace
155900 & HOSPITAL UNIVERSITARIO DA UFSCAR\\
155901 & HOSPITAL ESCOLA DA UFPEL\\
155902 & EBSERH CHC-UFPR\\
155903 & HOSPITAL UNIVERSITARIO DA UFJF\\
155904 & HOSPITAL DAS CLINICAS DA UFG\\
\addlinespace
155906 & MATERNIDADE CLIMERIO DE OLIVEIRA\\
155907 & HOSPITAL UNIVERSITARIO PROF. EDGARD SANTOS\\
155908 & HOSPITAL UNIV. DR. MIGUEL RIET CORREA JR.\\
155909 & COMPLEXO HOSP UNIVERSIT DA UFPA (HUBFS/HUJBB)\\
155910 & HOSPITAL UNIVERSITARIO DE LAGARTO\\
\addlinespace
155911 & HOSPITAL UNIVERSITARIO GAFFREE E GUINLE\\
155912 & HOSPITAL UNIVERSITARIO JULIO BANDEIRA\\
155913 & HOSPITAL UNIVERSITARIO DA UFSC\\
155914 & HOSPITAL UNIVERSITARIO ALCIDES CARNEIRO\\
155915 & HOSPITAL UNIVERSITARIO ANTONIO PEDRO\\
\addlinespace
156654 & HOSPITAL DE CLINICAS DE UBERLANDIA\\
154357 & HOSPITAL UNIVERSITARIO MARIA AP. PEDROSSIAM\\
154716 & HOSPITAL DE ENSINO DR WASHINGTON A. DE BARROS\\
156956 & HOSPITAL UNIVERSITARIO DA UNIV. FED. DO AMAPA\\
\bottomrule
\end{longtable}

\begin{Shaded}
\begin{Highlighting}[]
\NormalTok{SHA\_Uniao\_MEC\_acoes }\OtherTok{\textless{}{-}}\NormalTok{ df\_final }\SpecialCharTok{\%\textgreater{}\%}
  \FunctionTok{filter}\NormalTok{(UO\_Orgao\_Maximo }\SpecialCharTok{==} \StringTok{"MINISTERIO DA EDUCACAO"}\NormalTok{) }\SpecialCharTok{\%\textgreater{}\%} 
  \FunctionTok{select}\NormalTok{(}\StringTok{\textasciigrave{}}\AttributeTok{Ano Lançamento}\StringTok{\textasciigrave{}}\NormalTok{,UG\_Executora\_COD, UG\_Executora, Subfuncao\_COD, Acao\_Cod, Categoria\_Economica\_Despesa\_COD, }\StringTok{\textasciigrave{}}\AttributeTok{DESPESAS PAGAS}\StringTok{\textasciigrave{}}\NormalTok{, }\StringTok{\textasciigrave{}}\AttributeTok{RESTOS A PAGAR PROCESSADOS PAGOS}\StringTok{\textasciigrave{}}\NormalTok{, }\StringTok{\textasciigrave{}}\AttributeTok{RESTOS A PAGAR NAO PROCESSADOS PAGOS}\StringTok{\textasciigrave{}}\NormalTok{) }\SpecialCharTok{\%\textgreater{}\%} 
  \FunctionTok{filter}\NormalTok{(Categoria\_Economica\_Despesa\_COD }\SpecialCharTok{\%in\%} \StringTok{"3"}\NormalTok{) }\SpecialCharTok{\%\textgreater{}\%}
  \FunctionTok{filter}\NormalTok{(UG\_Executora\_COD }\SpecialCharTok{\%in\%}\NormalTok{ HUs\_}\SpecialCharTok{$}\NormalTok{UG\_Executora\_COD }\SpecialCharTok{|}\NormalTok{ Acao\_Cod }\SpecialCharTok{\%in\%} \FunctionTok{c}\NormalTok{(}\StringTok{"20RX"}\NormalTok{, }\StringTok{"4086"}\NormalTok{)) }\SpecialCharTok{\%\textgreater{}\%} 
  \FunctionTok{mutate\_at}\NormalTok{(}\FunctionTok{c}\NormalTok{(}\StringTok{"DESPESAS PAGAS"}\NormalTok{, }\StringTok{"RESTOS A PAGAR PROCESSADOS PAGOS"}\NormalTok{, }\StringTok{"RESTOS A PAGAR NAO PROCESSADOS PAGOS"}\NormalTok{), }\SpecialCharTok{\textasciitilde{}}\FunctionTok{replace\_na}\NormalTok{(.,}\DecValTok{0}\NormalTok{)) }\SpecialCharTok{\%\textgreater{}\%}
  \FunctionTok{mutate}\NormalTok{(}\AttributeTok{resultado =} \StringTok{\textasciigrave{}}\AttributeTok{DESPESAS PAGAS}\StringTok{\textasciigrave{}} \SpecialCharTok{+} \StringTok{\textasciigrave{}}\AttributeTok{RESTOS A PAGAR PROCESSADOS PAGOS}\StringTok{\textasciigrave{}} \SpecialCharTok{+} \StringTok{\textasciigrave{}}\AttributeTok{RESTOS A PAGAR NAO PROCESSADOS PAGOS}\StringTok{\textasciigrave{}}\NormalTok{) }\SpecialCharTok{\%\textgreater{}\%}
  \FunctionTok{select}\NormalTok{(}\StringTok{\textasciigrave{}}\AttributeTok{Ano Lançamento}\StringTok{\textasciigrave{}}\NormalTok{, resultado) }\SpecialCharTok{\%\textgreater{}\%}
  \FunctionTok{group\_by}\NormalTok{(}\StringTok{\textasciigrave{}}\AttributeTok{Ano Lançamento}\StringTok{\textasciigrave{}}\NormalTok{) }\SpecialCharTok{\%\textgreater{}\%}
  \FunctionTok{summarise}\NormalTok{(}\AttributeTok{resultado\_final\_mec =} \FunctionTok{round}\NormalTok{(}\FunctionTok{sum}\NormalTok{(resultado),}\DecValTok{2}\NormalTok{))}
\end{Highlighting}
\end{Shaded}

\hypertarget{calculando-a-gasto-do-ministuxe9rio-da-sauxfade}{%
\subsubsection{Calculando a gasto do Ministério da
Saúde}\label{calculando-a-gasto-do-ministuxe9rio-da-sauxfade}}

As principais despesas do Governo Federal com saúde são centralizadas no
Ministério da Saúde. Para estimar essas despesas, foram consideradas as
despesas correntes referentes (i) às Ações e Serviços Públicos de Saúde
(ASPS) pelo Ministério da Saúde; e (ii) às ações que não são ASPS, mas
que entram na perspectiva SHA, pelo Ministério da Saúde. Os dados foram
retirados do sistema Tesouro Gerencial, que possibilita a consulta das
informações do Sistema de Administração Financeira e Controle (SIAFI).

Para o cálculo de (i) e (ii), primeiramente, considerou-se o montante de
despesas da Função Saúde (Função 10) identificado como despesas
correntes (Categoria Econômica Despesa = 3) e com os filtros detalhados
no Quadro 1.

A fase de empenho utilizada é a soma de: Restos a Pagar Não Processados
Pagos + Restos a Pagar Não processados Pagos

\textbf{Quadro 1} - Composição do Gasto Federal do SUS pelo Ministério
da Saúde

\begin{longtable}[]{@{}
  >{\raggedright\arraybackslash}p{(\columnwidth - 2\tabcolsep) * \real{0.3351}}
  >{\raggedright\arraybackslash}p{(\columnwidth - 2\tabcolsep) * \real{0.6649}}@{}}
\toprule()
\begin{minipage}[b]{\linewidth}\raggedright
Despesas ASPS
\end{minipage} & \begin{minipage}[b]{\linewidth}\raggedright
Despesas Não ASPS
\end{minipage} \\
\midrule()
\endhead
Identificador de Uso para recursos destinados a identificação da
aplicação mínima em ações e serviços de saúde (Iduso=6) &
\begin{minipage}[t]{\linewidth}\raggedright
Ações programáticas:\\
• Assistência Médica e Odontológica aos Servidores Civis, Empregados,
Militares e seus Dependentes (2004) -- para as Unidades Orçamentárias do
Ministério da Saúde (FNS -- 36901), Fiocruz (36201), Funasa (36211), ANS
(36213) e Anvisa (36212)\\
• Reestruturação dos Serviços Ambulatoriais e Hospitalares Prestados
pelos Hospitais Universitários Federais (Financiamento Partilhado -
REHUF) (20G8)\\
• Manutenção e Funcionamento do Programa Farmácia Popular do Brasil pelo
Sistema de Co-pagamento (20YS)\\
• Apoio à Manutenção dos Polos de Academia da Saúde (217U)\\
• Estruturação de Academias da Saúde(20YL)\\

Unidades orçamentárias:\\
• ANVISA (36212), com exceção das despesas da Ação 2004\\
• ANS (36213), ), com exceção das despesas da Ação 2004\\

Fontes de recursos detalhada:\\
• Compensação Financeira pela produção de Petróleo,Gás Natural e Outros
(0142000000, 6142000000 e 6342000000)\\
• Parcela da Saúde -- Royalties 3ºart. 2ºl. Lei no. 12858/2013
(0142369010, 0342369010 e 6142369010)\\

Modalidade de aplicação\\
• Transferência a Instituições Multigovernamentais (70)\strut
\end{minipage} \\
\bottomrule()
\end{longtable}

\begin{Shaded}
\begin{Highlighting}[]
\NormalTok{SHA\_Uniao }\OtherTok{\textless{}{-}}\NormalTok{ df\_final }\SpecialCharTok{\%\textgreater{}\%} 
  \FunctionTok{filter}\NormalTok{(UO\_Orgao\_Maximo }\SpecialCharTok{==} \StringTok{"MINISTERIO DA SAUDE"}\NormalTok{) }\SpecialCharTok{\%\textgreater{}\%} 
  \FunctionTok{select}\NormalTok{(}\StringTok{\textasciigrave{}}\AttributeTok{Ano Lançamento}\StringTok{\textasciigrave{}}\NormalTok{, Fonte\_Recursos\_Detalhada\_COD, Iduso\_COD, Unidade\_Orcamentaria\_COD, Unidade\_Orcamentaria, UG\_Executora\_COD, UG\_Executora, Subfuncao\_COD, Acao\_Cod, Categoria\_Economica\_Despesa\_COD, Modalidade\_Aplicacao\_COD ,}\StringTok{\textasciigrave{}}\AttributeTok{DESPESAS PAGAS}\StringTok{\textasciigrave{}}\NormalTok{, }\StringTok{\textasciigrave{}}\AttributeTok{RESTOS A PAGAR PROCESSADOS PAGOS}\StringTok{\textasciigrave{}}\NormalTok{, }\StringTok{\textasciigrave{}}\AttributeTok{RESTOS A PAGAR NAO PROCESSADOS PAGOS}\StringTok{\textasciigrave{}}\NormalTok{) }\SpecialCharTok{\%\textgreater{}\%} 
  \FunctionTok{filter}\NormalTok{(Categoria\_Economica\_Despesa\_COD }\SpecialCharTok{\%in\%} \StringTok{"3"}\NormalTok{) }\SpecialCharTok{\%\textgreater{}\%} 
  \FunctionTok{filter}\NormalTok{(Iduso\_COD }\SpecialCharTok{==} \DecValTok{6} \SpecialCharTok{|}\NormalTok{ (Acao\_Cod }\SpecialCharTok{==} \StringTok{"2004"} \SpecialCharTok{\&}\NormalTok{ Unidade\_Orcamentaria\_COD }\SpecialCharTok{\%in\%} \FunctionTok{c}\NormalTok{(}\StringTok{"36901"}\NormalTok{, }\StringTok{"36211"}\NormalTok{, }\StringTok{"36213"}\NormalTok{, }\StringTok{"36212"}\NormalTok{)) }\SpecialCharTok{|}\NormalTok{ Acao\_Cod }\SpecialCharTok{\%in\%} \FunctionTok{c}\NormalTok{(}\StringTok{"20G8"}\NormalTok{, }\StringTok{"20YS"}\NormalTok{, }\StringTok{"217U"}\NormalTok{, }\StringTok{"20YL"}\NormalTok{) }\SpecialCharTok{|}\NormalTok{ Modalidade\_Aplicacao\_COD }\SpecialCharTok{==} \StringTok{"70"} \SpecialCharTok{|}\NormalTok{ Fonte\_Recursos\_Detalhada\_COD }\SpecialCharTok{\%in\%} \FunctionTok{c}\NormalTok{(}\StringTok{"0142000000"}\NormalTok{, }\StringTok{"6142000000"}\NormalTok{, }\StringTok{"6342000000"}\NormalTok{, }\StringTok{"0142369010"}\NormalTok{, }\StringTok{"0342369010"}\NormalTok{, }\StringTok{"6142369010"}\NormalTok{) }\SpecialCharTok{|}\NormalTok{ (Unidade\_Orcamentaria\_COD }\SpecialCharTok{\%in\%} \FunctionTok{c}\NormalTok{(}\StringTok{"36212"}\NormalTok{, }\StringTok{"36213"}\NormalTok{) }\SpecialCharTok{\&}\NormalTok{ Acao\_Cod }\SpecialCharTok{!=} \StringTok{"2004"}\NormalTok{)) }\SpecialCharTok{\%\textgreater{}\%} 
  \FunctionTok{mutate\_at}\NormalTok{(}\FunctionTok{c}\NormalTok{(}\StringTok{"DESPESAS PAGAS"}\NormalTok{, }\StringTok{"RESTOS A PAGAR PROCESSADOS PAGOS"}\NormalTok{, }\StringTok{"RESTOS A PAGAR NAO PROCESSADOS PAGOS"}\NormalTok{), }\SpecialCharTok{\textasciitilde{}}\FunctionTok{replace\_na}\NormalTok{(.,}\DecValTok{0}\NormalTok{)) }\SpecialCharTok{\%\textgreater{}\%}
  \FunctionTok{mutate}\NormalTok{(}\AttributeTok{resultado =} \StringTok{\textasciigrave{}}\AttributeTok{DESPESAS PAGAS}\StringTok{\textasciigrave{}} \SpecialCharTok{+} \StringTok{\textasciigrave{}}\AttributeTok{RESTOS A PAGAR PROCESSADOS PAGOS}\StringTok{\textasciigrave{}} \SpecialCharTok{+} \StringTok{\textasciigrave{}}\AttributeTok{RESTOS A PAGAR NAO PROCESSADOS PAGOS}\StringTok{\textasciigrave{}}\NormalTok{) }\SpecialCharTok{\%\textgreater{}\%}
  \FunctionTok{select}\NormalTok{(}\StringTok{\textasciigrave{}}\AttributeTok{Ano Lançamento}\StringTok{\textasciigrave{}}\NormalTok{, resultado) }\SpecialCharTok{\%\textgreater{}\%}
  \FunctionTok{group\_by}\NormalTok{(}\StringTok{\textasciigrave{}}\AttributeTok{Ano Lançamento}\StringTok{\textasciigrave{}}\NormalTok{) }\SpecialCharTok{\%\textgreater{}\%}
  \FunctionTok{summarise}\NormalTok{(}\AttributeTok{resultado\_final\_uniao\_sus =} \FunctionTok{round}\NormalTok{(}\FunctionTok{sum}\NormalTok{(resultado),}\DecValTok{2}\NormalTok{))}
\end{Highlighting}
\end{Shaded}

\hypertarget{calculando-a-gasto-dos-entes-subnacionais}{%
\subsubsection{Calculando a gasto dos entes
subnacionais}\label{calculando-a-gasto-dos-entes-subnacionais}}

As despesas aqui consideradas são aquelas realizadas por Estados,
Distrito Federal e Municípios. São consideradas as despesas correntes em
ASPS\footnote{(Despesas correntes) - (Inativos e Pensionistas) -
  (Despesas correntes com outras ações e serviços não
  computados\uline{)}~= \textbf{Despesas correntes em ASPS}} e ao
contrário das estimativas para a União, utiliza-se a fase de liquidação
das despesas\footnote{A publicação do SHA de 2022, que trata sobre as
  despesas de 2015-2019, especifica que a fase de liquidação permite
  aproximação melhor do gasto real de Estados e Municípios}.

\hypertarget{estados}{%
\paragraph{Estados}\label{estados}}

\begin{Shaded}
\begin{Highlighting}[]
\CommentTok{\#Importando a base {-}{-}{-}{-}{-}{-}{-}{-}{-}{-}{-}{-}{-}{-}{-}{-}{-}{-}{-}{-}{-}{-}{-}{-}{-}{-}}
\CommentTok{\# Diretório onde estão os arquivos xlsx}
\NormalTok{diretorio }\OtherTok{\textless{}{-}} \FunctionTok{getwd}\NormalTok{()}
\NormalTok{dir\_dados }\OtherTok{\textless{}{-}} \FunctionTok{file.path}\NormalTok{(diretorio, }\StringTok{"dados\_estados\_munics"}\NormalTok{)}

\CommentTok{\# Listar os arquivos no diretório de dados}
\NormalTok{arquivos }\OtherTok{\textless{}{-}} \FunctionTok{list.files}\NormalTok{(}\AttributeTok{path =}\NormalTok{ dir\_dados, }\AttributeTok{pattern =} \StringTok{"estadual"}\NormalTok{, }\AttributeTok{full.names =} \ConstantTok{TRUE}\NormalTok{)}
\NormalTok{arquivo\_estadual }\OtherTok{\textless{}{-}}\NormalTok{ arquivos[}\DecValTok{1}\NormalTok{]}

\CommentTok{\# Importar as abas específicas como data frames separados}
\FunctionTok{suppressMessages}\NormalTok{(df\_2021 }\OtherTok{\textless{}{-}} \FunctionTok{read\_excel}\NormalTok{(arquivo\_estadual, }\AttributeTok{sheet =} \StringTok{"2021"}\NormalTok{))}
\FunctionTok{suppressMessages}\NormalTok{(df\_2022 }\OtherTok{\textless{}{-}} \FunctionTok{read\_excel}\NormalTok{(arquivo\_estadual, }\AttributeTok{sheet =} \StringTok{"2022"}\NormalTok{))}

\CommentTok{\#Fazendo as modificações necessárias nos DFs{-}{-}{-}{-}{-}{-}{-}{-}}
\CommentTok{\# Definir uma função para renomear as colunas}
\NormalTok{modificar\_dataframe }\OtherTok{\textless{}{-}} \ControlFlowTok{function}\NormalTok{(df, ano) \{}
  \CommentTok{\# Renomear as primeiras três colunas}
  \FunctionTok{names}\NormalTok{(df)[}\DecValTok{1}\NormalTok{] }\OtherTok{\textless{}{-}} \StringTok{\textquotesingle{}cod\_UF\textquotesingle{}}
  \FunctionTok{names}\NormalTok{(df)[}\DecValTok{2}\NormalTok{] }\OtherTok{\textless{}{-}} \StringTok{\textquotesingle{}UF\textquotesingle{}}
  \FunctionTok{names}\NormalTok{(df)[}\DecValTok{3}\NormalTok{] }\OtherTok{\textless{}{-}} \StringTok{\textquotesingle{}nome\_UF\textquotesingle{}}
  
  \CommentTok{\# Obter os nomes das colunas a partir da quarta coluna}
\NormalTok{  vecnames }\OtherTok{\textless{}{-}} \FunctionTok{names}\NormalTok{(df)[}\SpecialCharTok{{-}}\NormalTok{(}\DecValTok{1}\SpecialCharTok{:}\DecValTok{3}\NormalTok{)]}
  
  \CommentTok{\# Substituir nomes que contêm números de 1 a 9 por NA}
\NormalTok{  vecnames[}\FunctionTok{grepl}\NormalTok{(}\StringTok{\textquotesingle{}[1{-}9]\textquotesingle{}}\NormalTok{, vecnames)] }\OtherTok{\textless{}{-}} \ConstantTok{NA}
  
  \CommentTok{\# Preencher valores NA com o último valor não{-}NA à esquerda}
\NormalTok{  vecnames }\OtherTok{\textless{}{-}} \FunctionTok{na.locf}\NormalTok{(vecnames)}
  
  \CommentTok{\# Obter os nomes originais das colunas para criar vecnames3}
\NormalTok{  vecnames2 }\OtherTok{\textless{}{-}}\NormalTok{ df[}\DecValTok{1}\NormalTok{, }\SpecialCharTok{{-}}\NormalTok{(}\DecValTok{1}\SpecialCharTok{:}\DecValTok{3}\NormalTok{)]}
\NormalTok{  vecnames3 }\OtherTok{\textless{}{-}} \FunctionTok{paste0}\NormalTok{(vecnames, }\StringTok{\textquotesingle{}\_\textquotesingle{}}\NormalTok{, vecnames2)}
  
  \CommentTok{\# Atualizar os nomes das colunas a partir da quarta coluna}
  \FunctionTok{names}\NormalTok{(df)[}\SpecialCharTok{{-}}\NormalTok{(}\DecValTok{1}\SpecialCharTok{:}\DecValTok{3}\NormalTok{)] }\OtherTok{\textless{}{-}}\NormalTok{ vecnames3}
  
  \CommentTok{\# Remover a primeira linha (que foi usada para obter vecnames2)}
\NormalTok{  df }\OtherTok{\textless{}{-}}\NormalTok{ df[}\SpecialCharTok{{-}}\DecValTok{1}\NormalTok{, ]}
  
  \CommentTok{\# Transformar o data frame de formato largo para formato longo}
\NormalTok{  melt.data }\OtherTok{\textless{}{-}} \FunctionTok{melt}\NormalTok{(df, }\AttributeTok{id.vars =} \FunctionTok{c}\NormalTok{(}\StringTok{\textquotesingle{}cod\_UF\textquotesingle{}}\NormalTok{, }\StringTok{\textquotesingle{}UF\textquotesingle{}}\NormalTok{, }\StringTok{\textquotesingle{}nome\_UF\textquotesingle{}}\NormalTok{))}
  
  \CommentTok{\# Converter a coluna \textquotesingle{}variable\textquotesingle{} para caractere}
\NormalTok{  melt.data}\SpecialCharTok{$}\NormalTok{variable }\OtherTok{\textless{}{-}} \FunctionTok{as.character}\NormalTok{(melt.data}\SpecialCharTok{$}\NormalTok{variable)}
  
  \CommentTok{\# Separar a coluna \textquotesingle{}variable\textquotesingle{} em \textquotesingle{}Fase\_Despesa\textquotesingle{} e \textquotesingle{}Natureza\textquotesingle{}}
\NormalTok{  melt.data }\OtherTok{\textless{}{-}}\NormalTok{ melt.data }\SpecialCharTok{\%\textgreater{}\%}
    \FunctionTok{separate}\NormalTok{(variable, }\FunctionTok{c}\NormalTok{(}\StringTok{\textquotesingle{}Fase\_Despesa\textquotesingle{}}\NormalTok{, }\StringTok{\textquotesingle{}Natureza\textquotesingle{}}\NormalTok{), }\StringTok{\textquotesingle{}\_\textquotesingle{}}\NormalTok{)}
  
  \CommentTok{\# Criar a coluna \textquotesingle{}ano\textquotesingle{}}
\NormalTok{  melt.data}\SpecialCharTok{$}\NormalTok{ano }\OtherTok{\textless{}{-}}\NormalTok{ ano}
  
  \CommentTok{\# Converter a coluna \textquotesingle{}value\textquotesingle{} para numérico}
\NormalTok{  melt.data}\SpecialCharTok{$}\NormalTok{value }\OtherTok{\textless{}{-}} \FunctionTok{as.numeric}\NormalTok{(melt.data}\SpecialCharTok{$}\NormalTok{value)}
  
  \CommentTok{\# Remover acentos da coluna \textquotesingle{}nome\_UF\textquotesingle{}}
\NormalTok{  melt.data}\SpecialCharTok{$}\NormalTok{nome\_UF }\OtherTok{\textless{}{-}} \FunctionTok{iconv}\NormalTok{(melt.data}\SpecialCharTok{$}\NormalTok{nome\_UF, }\AttributeTok{from =} \StringTok{\textquotesingle{}UTF{-}8\textquotesingle{}}\NormalTok{, }\AttributeTok{to =} \StringTok{\textquotesingle{}ASCII//TRANSLIT\textquotesingle{}}\NormalTok{)}
  
  \FunctionTok{return}\NormalTok{(melt.data)}
\NormalTok{\}}
\CommentTok{\# Lista de data frames}
\NormalTok{dfs }\OtherTok{\textless{}{-}} \FunctionTok{list}\NormalTok{(df\_2021, df\_2022)}
\NormalTok{anos }\OtherTok{\textless{}{-}} \FunctionTok{c}\NormalTok{(}\StringTok{"2021"}\NormalTok{, }\StringTok{"2022"}\NormalTok{)}

\CommentTok{\# Aplicar a função de renomeação a cada data frame na lista}
\NormalTok{dfs\_modificados }\OtherTok{\textless{}{-}} \FunctionTok{mapply}\NormalTok{(modificar\_dataframe, dfs, anos, }\AttributeTok{SIMPLIFY =}\NormalTok{ F)}

\CommentTok{\# Separar os data frames modificados de volta para as variáveis individuais}
\NormalTok{df\_2021 }\OtherTok{\textless{}{-}}\NormalTok{ dfs\_modificados[[}\DecValTok{1}\NormalTok{]]}
\NormalTok{df\_2022 }\OtherTok{\textless{}{-}}\NormalTok{ dfs\_modificados[[}\DecValTok{2}\NormalTok{]]}

\NormalTok{df\_2021\_2022 }\OtherTok{\textless{}{-}} \FunctionTok{bind\_rows}\NormalTok{(df\_2021, df\_2022)}

\CommentTok{\#Calculando a Despesa total corrente em ASPS dos Estados{-}{-}{-}{-}{-}{-}{-}{-}{-}}
\NormalTok{resultado\_estadual }\OtherTok{\textless{}{-}}\NormalTok{ df\_2021\_2022 }\SpecialCharTok{\%\textgreater{}\%} 
  \FunctionTok{filter}\NormalTok{(Natureza }\SpecialCharTok{==} \StringTok{"Despesas correntes em ASPS"}\NormalTok{) }\SpecialCharTok{\%\textgreater{}\%} 
  \FunctionTok{group\_by}\NormalTok{(ano) }\SpecialCharTok{\%\textgreater{}\%} 
  \FunctionTok{summarise}\NormalTok{(}\AttributeTok{total\_ufs =} \FunctionTok{round}\NormalTok{(}\FunctionTok{sum}\NormalTok{(value)}\SpecialCharTok{/}\DecValTok{10}\SpecialCharTok{\^{}}\DecValTok{9}\NormalTok{,}\DecValTok{2}\NormalTok{))}
\end{Highlighting}
\end{Shaded}

\hypertarget{municuxedpios}{%
\paragraph{Municípios}\label{municuxedpios}}

\begin{Shaded}
\begin{Highlighting}[]
\CommentTok{\#Importando a base {-}{-}{-}{-}{-}{-}{-}{-}{-}{-}{-}{-}{-}{-}{-}{-}{-}{-}{-}{-}{-}{-}{-}{-}{-}{-}}
\CommentTok{\# Diretório onde estão os arquivos xlsx}
\NormalTok{diretorio }\OtherTok{\textless{}{-}} \FunctionTok{getwd}\NormalTok{()}
\NormalTok{dir\_dados }\OtherTok{\textless{}{-}} \FunctionTok{file.path}\NormalTok{(diretorio, }\StringTok{"dados\_estados\_munics"}\NormalTok{)}

\CommentTok{\# Listar os arquivos no diretório de dados}
\NormalTok{arquivos }\OtherTok{\textless{}{-}} \FunctionTok{list.files}\NormalTok{(}\AttributeTok{path =}\NormalTok{ dir\_dados, }\AttributeTok{pattern =} \StringTok{"munic"}\NormalTok{, }\AttributeTok{full.names =} \ConstantTok{TRUE}\NormalTok{)}
\NormalTok{arquivo\_munic }\OtherTok{\textless{}{-}}\NormalTok{ arquivos[}\DecValTok{1}\NormalTok{]}

\CommentTok{\# Importar as abas específicas como data frames separados}
\FunctionTok{suppressMessages}\NormalTok{(df\_2021\_munic }\OtherTok{\textless{}{-}} \FunctionTok{read\_excel}\NormalTok{(arquivo\_munic, }\AttributeTok{sheet =} \StringTok{"2021"}\NormalTok{))}
\FunctionTok{suppressMessages}\NormalTok{(df\_2022\_munic }\OtherTok{\textless{}{-}} \FunctionTok{read\_excel}\NormalTok{(arquivo\_munic, }\AttributeTok{sheet =} \StringTok{"2022"}\NormalTok{))}

\CommentTok{\#Fazendo as modificações necessárias nos DFs{-}{-}{-}{-}{-}{-}{-}{-}}
\CommentTok{\# Definir uma função para renomear as colunas}
\NormalTok{modificar\_dataframe\_munic }\OtherTok{\textless{}{-}} \ControlFlowTok{function}\NormalTok{(df, ano) \{}
  \CommentTok{\# Renomear as primeiras três colunas}
  \FunctionTok{names}\NormalTok{(df)[}\DecValTok{1}\NormalTok{] }\OtherTok{\textless{}{-}} \StringTok{\textquotesingle{}cod\_munic\textquotesingle{}}
  \FunctionTok{names}\NormalTok{(df)[}\DecValTok{2}\NormalTok{] }\OtherTok{\textless{}{-}} \StringTok{\textquotesingle{}Munic\textquotesingle{}}
  \FunctionTok{names}\NormalTok{(df)[}\DecValTok{3}\NormalTok{] }\OtherTok{\textless{}{-}} \StringTok{\textquotesingle{}UF\textquotesingle{}}
  
  \CommentTok{\# Obter os nomes das colunas a partir da quarta coluna}
\NormalTok{  vecnames }\OtherTok{\textless{}{-}} \FunctionTok{names}\NormalTok{(df)[}\SpecialCharTok{{-}}\NormalTok{(}\DecValTok{1}\SpecialCharTok{:}\DecValTok{3}\NormalTok{)]}
  
  \CommentTok{\# Substituir nomes que contêm números de 1 a 9 por NA}
\NormalTok{  vecnames[}\FunctionTok{grepl}\NormalTok{(}\StringTok{\textquotesingle{}[1{-}9]\textquotesingle{}}\NormalTok{, vecnames)] }\OtherTok{\textless{}{-}} \ConstantTok{NA}
  
  \CommentTok{\# Preencher valores NA com o último valor não{-}NA à esquerda}
\NormalTok{  vecnames }\OtherTok{\textless{}{-}} \FunctionTok{na.locf}\NormalTok{(vecnames)}
  
  \CommentTok{\# Obter os nomes originais das colunas para criar vecnames3}
\NormalTok{  vecnames2 }\OtherTok{\textless{}{-}}\NormalTok{ df[}\DecValTok{1}\NormalTok{, }\SpecialCharTok{{-}}\NormalTok{(}\DecValTok{1}\SpecialCharTok{:}\DecValTok{3}\NormalTok{)]}
\NormalTok{  vecnames3 }\OtherTok{\textless{}{-}} \FunctionTok{paste0}\NormalTok{(vecnames, }\StringTok{\textquotesingle{}\_\textquotesingle{}}\NormalTok{, vecnames2)}
  
  \CommentTok{\# Atualizar os nomes das colunas a partir da quarta coluna}
  \FunctionTok{names}\NormalTok{(df)[}\SpecialCharTok{{-}}\NormalTok{(}\DecValTok{1}\SpecialCharTok{:}\DecValTok{3}\NormalTok{)] }\OtherTok{\textless{}{-}}\NormalTok{ vecnames3}
  
  \CommentTok{\# Remover a primeira linha (que foi usada para obter vecnames2)}
\NormalTok{  df }\OtherTok{\textless{}{-}}\NormalTok{ df[}\SpecialCharTok{{-}}\DecValTok{1}\NormalTok{, ]}
  
  \CommentTok{\# Transformar o data frame de formato largo para formato longo}
\NormalTok{  melt.data }\OtherTok{\textless{}{-}} \FunctionTok{melt}\NormalTok{(df, }\AttributeTok{id.vars =} \FunctionTok{c}\NormalTok{(}\StringTok{\textquotesingle{}cod\_munic\textquotesingle{}}\NormalTok{, }\StringTok{\textquotesingle{}Munic\textquotesingle{}}\NormalTok{, }\StringTok{\textquotesingle{}UF\textquotesingle{}}\NormalTok{))}
  
  \CommentTok{\# Converter a coluna \textquotesingle{}variable\textquotesingle{} para caractere}
\NormalTok{  melt.data}\SpecialCharTok{$}\NormalTok{variable }\OtherTok{\textless{}{-}} \FunctionTok{as.character}\NormalTok{(melt.data}\SpecialCharTok{$}\NormalTok{variable)}
  
  \CommentTok{\# Separar a coluna \textquotesingle{}variable\textquotesingle{} em \textquotesingle{}Fase\_Despesa\textquotesingle{} e \textquotesingle{}Natureza\textquotesingle{}}
\NormalTok{  melt.data }\OtherTok{\textless{}{-}}\NormalTok{ melt.data }\SpecialCharTok{\%\textgreater{}\%}
    \FunctionTok{separate}\NormalTok{(variable, }\FunctionTok{c}\NormalTok{(}\StringTok{\textquotesingle{}Fase\_Despesa\textquotesingle{}}\NormalTok{, }\StringTok{\textquotesingle{}Natureza\textquotesingle{}}\NormalTok{), }\StringTok{\textquotesingle{}\_\textquotesingle{}}\NormalTok{)}
  
  \CommentTok{\# Criar a coluna \textquotesingle{}ano\textquotesingle{}}
\NormalTok{  melt.data}\SpecialCharTok{$}\NormalTok{ano }\OtherTok{\textless{}{-}}\NormalTok{ ano}
  
  \CommentTok{\# Converter a coluna \textquotesingle{}value\textquotesingle{} para numérico}
\NormalTok{  melt.data}\SpecialCharTok{$}\NormalTok{value }\OtherTok{\textless{}{-}} \FunctionTok{as.numeric}\NormalTok{(melt.data}\SpecialCharTok{$}\NormalTok{value)}
  
  \CommentTok{\# Remover acentos da coluna \textquotesingle{}nome\_UF\textquotesingle{}}
\NormalTok{  melt.data}\SpecialCharTok{$}\NormalTok{Munic }\OtherTok{\textless{}{-}} \FunctionTok{iconv}\NormalTok{(melt.data}\SpecialCharTok{$}\NormalTok{Munic, }\AttributeTok{from =} \StringTok{\textquotesingle{}UTF{-}8\textquotesingle{}}\NormalTok{, }\AttributeTok{to =} \StringTok{\textquotesingle{}ASCII//TRANSLIT\textquotesingle{}}\NormalTok{)}
  
  \FunctionTok{return}\NormalTok{(melt.data)}
\NormalTok{\}}
\CommentTok{\# Lista de data frames}
\NormalTok{dfs\_munic }\OtherTok{\textless{}{-}} \FunctionTok{list}\NormalTok{(df\_2021\_munic, df\_2022\_munic)}
\NormalTok{anos }\OtherTok{\textless{}{-}} \FunctionTok{c}\NormalTok{(}\StringTok{"2021"}\NormalTok{, }\StringTok{"2022"}\NormalTok{)}

\CommentTok{\# Aplicar a função de renomeação a cada data frame na lista}
\NormalTok{dfs\_mounics }\OtherTok{\textless{}{-}} \FunctionTok{mapply}\NormalTok{(modificar\_dataframe\_munic, dfs\_munic, anos, }\AttributeTok{SIMPLIFY =}\NormalTok{ F)}

\CommentTok{\# Separar os data frames modificados de volta para as variáveis individuais}
\NormalTok{df\_2021\_munic }\OtherTok{\textless{}{-}}\NormalTok{ dfs\_mounics[[}\DecValTok{1}\NormalTok{]]}
\NormalTok{df\_2022\_munic }\OtherTok{\textless{}{-}}\NormalTok{ dfs\_mounics[[}\DecValTok{2}\NormalTok{]]}

\NormalTok{df\_2021\_2022\_munic }\OtherTok{\textless{}{-}} \FunctionTok{bind\_rows}\NormalTok{(df\_2021\_munic, df\_2022\_munic)}

\CommentTok{\#Calculando a Despesa total corrente em ASPS dos Estados{-}{-}{-}{-}{-}{-}{-}{-}{-}}
\NormalTok{resultado\_municipal }\OtherTok{\textless{}{-}}\NormalTok{ df\_2021\_2022\_munic }\SpecialCharTok{\%\textgreater{}\%} 
  \FunctionTok{filter}\NormalTok{(Natureza }\SpecialCharTok{==} \StringTok{"Despesas correntes em ASPS"}\NormalTok{) }\SpecialCharTok{\%\textgreater{}\%} 
  \FunctionTok{group\_by}\NormalTok{(ano) }\SpecialCharTok{\%\textgreater{}\%} 
  \FunctionTok{summarise}\NormalTok{(}\AttributeTok{total\_munics =} \FunctionTok{round}\NormalTok{(}\FunctionTok{sum}\NormalTok{(value)}\SpecialCharTok{/}\DecValTok{10}\SpecialCharTok{\^{}}\DecValTok{9}\NormalTok{,}\DecValTok{2}\NormalTok{))}
\end{Highlighting}
\end{Shaded}

\hypertarget{resultados}{%
\subsection{Resultados}\label{resultados}}

A Tabela 2 mostra os valores financiados pela União (Ministério da Saúde
e Hospitais Universitários Federais).

\begin{Shaded}
\begin{Highlighting}[]
\NormalTok{SHA\_Uniao }\OtherTok{\textless{}{-}}\NormalTok{ SHA\_Uniao }\SpecialCharTok{\%\textgreater{}\%} 
  \FunctionTok{rename}\NormalTok{(}\StringTok{"ano"} \OtherTok{=} \StringTok{\textquotesingle{}Ano Lançamento\textquotesingle{}}\NormalTok{) }\SpecialCharTok{\%\textgreater{}\%} 
  \FunctionTok{mutate}\NormalTok{(}\AttributeTok{resultado\_final\_uniao\_sus =} \FunctionTok{round}\NormalTok{(resultado\_final\_uniao\_sus}\SpecialCharTok{/}\DecValTok{10}\SpecialCharTok{\^{}}\DecValTok{9}\NormalTok{, }\DecValTok{2}\NormalTok{))}

\NormalTok{SHA\_Uniao\_MEC\_acoes }\OtherTok{\textless{}{-}}\NormalTok{ SHA\_Uniao\_MEC\_acoes }\SpecialCharTok{\%\textgreater{}\%} 
  \FunctionTok{rename}\NormalTok{(}\StringTok{"ano"} \OtherTok{=} \StringTok{\textquotesingle{}Ano Lançamento\textquotesingle{}}\NormalTok{) }\SpecialCharTok{\%\textgreater{}\%} 
  \FunctionTok{mutate}\NormalTok{(}\AttributeTok{resultado\_final\_mec =} \FunctionTok{round}\NormalTok{(resultado\_final\_mec}\SpecialCharTok{/}\DecValTok{10}\SpecialCharTok{\^{}}\DecValTok{9}\NormalTok{,}\DecValTok{2}\NormalTok{))}

\NormalTok{resultado }\OtherTok{\textless{}{-}} \FunctionTok{cbind}\NormalTok{(SHA\_Uniao, SHA\_Uniao\_MEC\_acoes, resultado\_estadual, resultado\_municipal)}

\NormalTok{resultado }\OtherTok{\textless{}{-}}\NormalTok{ resultado[, }\FunctionTok{c}\NormalTok{(}\DecValTok{1}\NormalTok{, }\DecValTok{2}\NormalTok{, }\DecValTok{4}\NormalTok{, }\DecValTok{6}\NormalTok{, }\DecValTok{8}\NormalTok{)]}
\NormalTok{resultado}\SpecialCharTok{$}\NormalTok{total }\OtherTok{\textless{}{-}} \FunctionTok{rowSums}\NormalTok{(resultado[,}\DecValTok{2}\SpecialCharTok{:}\DecValTok{5}\NormalTok{])}

\NormalTok{resultado\_renamed }\OtherTok{\textless{}{-}}\NormalTok{ resultado }\SpecialCharTok{\%\textgreater{}\%} 
  \FunctionTok{rename}\NormalTok{(}\StringTok{"Ano"} \OtherTok{=}\NormalTok{ ano, }\StringTok{"MS"} \OtherTok{=}\NormalTok{ resultado\_final\_uniao\_sus, }\StringTok{"MEC"} \OtherTok{=}\NormalTok{ resultado\_final\_mec, }\StringTok{"Total Estados"} \OtherTok{=}\NormalTok{ total\_ufs, }\StringTok{"Total Municípios"} \OtherTok{=}\NormalTok{ total\_munics ,}\StringTok{"Total"} \OtherTok{=}\NormalTok{ total)}

\FunctionTok{kable}\NormalTok{(resultado\_renamed, }\AttributeTok{caption =} \StringTok{"Tabela 2 {-} Estimativas Despesas Sistema Único de Saúde (HF.1.1.1)"}\NormalTok{) }\SpecialCharTok{\%\textgreater{}\%}
  \FunctionTok{kable\_styling}\NormalTok{(}\AttributeTok{bootstrap\_options =} \FunctionTok{c}\NormalTok{(}\StringTok{"striped"}\NormalTok{, }\StringTok{"hover"}\NormalTok{, }\StringTok{"condensed"}\NormalTok{, }\StringTok{"responsive"}\NormalTok{), }
                \AttributeTok{full\_width =}\NormalTok{ F, }
                \AttributeTok{position =} \StringTok{"center"}\NormalTok{)}
\end{Highlighting}
\end{Shaded}

\begin{longtable}[t]{lrrrrr}
\caption{Tabela 2 - Estimativas Despesas Sistema Único de Saúde (HF.1.1.1)}\\
\toprule
Ano & MS & MEC & Total Estados & Total Municípios & Total\\
\midrule
2021 & 170.68 & 11.52 & 88.86 & 114.39 & 385.45\\
2022 & 149.98 & 11.47 & 101.52 & 134.41 & 397.38\\
\bottomrule
\end{longtable}

Dessa forma, sem a realização de ajustes, em 2022, as despesas estimadas
do SUS somaram R\$ 397,38 bi, sendo R\$ 149,98 bi provenientes do
Ministério da Saúde, R\$ 11,47 bi provenientes dos Hospitais
Universitários Federais, além de R\$ 101,52 bi dos Estados e Distrito
Federal, e R\$ 134,41 bi dos Municípios.

\begin{Shaded}
\begin{Highlighting}[]
\NormalTok{dados\_grafico }\OtherTok{\textless{}{-}}\NormalTok{ resultado\_renamed }\SpecialCharTok{\%\textgreater{}\%}
  \FunctionTok{pivot\_longer}\NormalTok{(}\AttributeTok{cols =} \FunctionTok{c}\NormalTok{(MS, MEC, }\StringTok{\textasciigrave{}}\AttributeTok{Total Estados}\StringTok{\textasciigrave{}}\NormalTok{, }\StringTok{\textasciigrave{}}\AttributeTok{Total Municípios}\StringTok{\textasciigrave{}}\NormalTok{), }
               \AttributeTok{names\_to =} \StringTok{"Parte"}\NormalTok{, }
               \AttributeTok{values\_to =} \StringTok{"Valor"}\NormalTok{) }\SpecialCharTok{\%\textgreater{}\%}
  \FunctionTok{group\_by}\NormalTok{(Ano) }\SpecialCharTok{\%\textgreater{}\%}
  \FunctionTok{mutate}\NormalTok{(}\AttributeTok{Total\_ano =} \FunctionTok{sum}\NormalTok{(Valor),}
         \AttributeTok{Share =}\NormalTok{ Valor }\SpecialCharTok{/}\NormalTok{ Total\_ano }\SpecialCharTok{*} \DecValTok{100}\NormalTok{) }\SpecialCharTok{\%\textgreater{}\%}
  \FunctionTok{ungroup}\NormalTok{()}

\CommentTok{\# Criar o gráfico}
\NormalTok{cores\_pastel }\OtherTok{\textless{}{-}} \FunctionTok{c}\NormalTok{(}\StringTok{"MS"} \OtherTok{=} \StringTok{"\#FFB3BA"}\NormalTok{, }\StringTok{"MEC"} \OtherTok{=} \StringTok{"\#FFDFBA"}\NormalTok{, }\StringTok{"Total Estadual"} \OtherTok{=} \StringTok{"\#FFFFBA"}\NormalTok{, }\StringTok{"Total Municípios"} \OtherTok{=} \StringTok{"\#BAFFC9"}\NormalTok{)}

\NormalTok{plot\_participacao }\OtherTok{\textless{}{-}} \FunctionTok{ggplot}\NormalTok{(dados\_grafico, }\FunctionTok{aes}\NormalTok{(}\AttributeTok{x =} \FunctionTok{factor}\NormalTok{(Ano), }\AttributeTok{y =}\NormalTok{ Share, }\AttributeTok{fill =}\NormalTok{ Parte)) }\SpecialCharTok{+}
  \FunctionTok{geom\_bar}\NormalTok{(}\AttributeTok{stat =} \StringTok{"identity"}\NormalTok{, }\AttributeTok{position =} \StringTok{"stack"}\NormalTok{) }\SpecialCharTok{+}
  \FunctionTok{geom\_text}\NormalTok{(}\FunctionTok{aes}\NormalTok{(}\AttributeTok{label =} \FunctionTok{sprintf}\NormalTok{(}\StringTok{"\%.1f\%\%"}\NormalTok{, Share)), }
            \AttributeTok{position =} \FunctionTok{position\_stack}\NormalTok{(}\AttributeTok{vjust =} \FloatTok{0.5}\NormalTok{), }\AttributeTok{size =} \DecValTok{3}\NormalTok{) }\SpecialCharTok{+}
  \FunctionTok{labs}\NormalTok{(}\AttributeTok{title =} \StringTok{"Gráfico 1 {-} Participação relativa na composição do gasto SUS (HF.1.1.1)"}\NormalTok{,}
       \AttributeTok{x =} \StringTok{"Ano"}\NormalTok{,}
       \AttributeTok{y =} \StringTok{"Participação (\%)"}\NormalTok{,}
       \AttributeTok{fill =} \StringTok{"Parte"}\NormalTok{) }\SpecialCharTok{+}
  \FunctionTok{scale\_fill\_manual}\NormalTok{(}\AttributeTok{values =}\NormalTok{ cores\_pastel) }\SpecialCharTok{+}
  \FunctionTok{theme\_minimal}\NormalTok{()}

\FunctionTok{print}\NormalTok{(plot\_participacao)}
\end{Highlighting}
\end{Shaded}

\begin{figure}[H]

{\centering \includegraphics{relatorio_hf_sus_files/figure-pdf/unnamed-chunk-10-1.pdf}

}

\end{figure}

Identificou-se um aumento das despesas em relação a 2021 da magnitude de
3,09\%, puxada principalmente pelo aumento das despesas dos entes
subnacionais, como mostra o Gráfico 1. No caso da União, observa-se uma
queda de 12,12\%, decorrente principalmente da diminuição dos créditos
extraordinários da Covid-19.



\end{document}
